\usepackage[T1]{fontenc}
\usepackage[utf8]{inputenc}
\usepackage[icelandic]{babel}
\usepackage{titlesec}
\usepackage{amsmath,amsfonts,amsthm}
\usepackage{mathtools}
\usepackage{fourier}
\usepackage{fullpage}
\usepackage{graphicx}
\usepackage{subcaption}
\usepackage{pgf,tikz,pgfplots}
\pgfplotsset{compat=newest}
\usepackage{mathrsfs}
\usetikzlibrary{arrows,backgrounds,calc,intersections,through}

\theoremstyle{definition}
\newtheorem*{definition}{Skilgreining}

\theoremstyle{plain}
\newtheorem{proposition}{Verkefni}

\theoremstyle{plain}
\newtheorem{theorem}{Setning}

\newtheoremstyle{examplesty}         % Name
            {}                   % Above skip 
            {}                   % Below skip
            {\upshape}           % Body font
            {}                   % Indent
            {\bfseries} % Head font
            {}                   % Head body punct
            {1em}                % Space after head
            {}                   % Heading

% Examples rejig the theorem environment
\theoremstyle{examplesty}
\newtheorem{example}{Dæmi}

% Solutions use a modified proof environment
\newenvironment{solution}
               {\let\oldqedsymbol=\qedsymbol
                \renewcommand{\qedsymbol}{}
                \begin{proof}[\bfseries\upshape Lausn]}
               {\end{proof}
                \renewcommand{\qedsymbol}{\oldqedsymbol}}

\theoremstyle{remark}
\newtheorem*{remark}{Athugasemd}

\titleformat{\section}[hang]
{\normalfont\bfseries}
{}{0.5em}{}

\titleformat{\subsection}[wrap]
{\normalfont\bfseries}
{}{0.5em}{}

%Common number systems:
\newcommand{\R}{\mathbb{R}}
\newcommand{\Q}{\mathbb{Q}}
\newcommand{\Z}{\mathbb{Z}}
\newcommand{\N}{\mathbb{N}}

%Straight line:
\newcommand{\stline}[1]{\langle #1 \rangle}
%Ray to the right:
\newcommand{\ray}[1]{[ #1 \rangle}
%line segment:
\newcommand{\linseg}[1]{\left[ #1\right]}
%length:
\newcommand{\len}[1]{\left| [#1] \right|}
%arc:
\newcommand{\arc}[1]{\widearc{#1}}

%Some common functions:
\newcommand{\sign}[1]{\operatorname{sign}\left( #1 \right)}
\newcommand{\floor}[1]{\left\lfloor #1 \right\rfloor}
