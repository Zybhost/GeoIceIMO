\section{Frumreglur flatarmyndarfræði}

Í þessum kafla setjum við fram frumsendur Evklíðskar rúmfræði. Upphaflega setti Evklíð fram fimm frumsendur fyrir flatarmyndarfræði og sannar út frá þeim rúmfræðilegar setningar. Var hann þannig fyrstur til setja fram frumreglur og leiða svo út frá þeim stærðfræðilegar setningar. \ldots

Seinna kom í ljós að Evklíð nýtti sér ýmsar staðreyndir sem hann getur ekki til í frumsendunum. \ldots Hér þarf að leita sögulegra staðreynda.

Þess ber þó að geta að Evklíð áttaði sig á samsíðafrumsendan væri óháð hinum frumsendunum, þ.e. ekki væri unnt að leiða hana út frá þeim. Öldum saman reyndu ýmsir við að sanna samsíðafrumsenduna út frá hinum en varð lítið ágengt. Það var svo á endanum á 18. öld HÉR ÞARF AÐ LEITA HEIMILDA sem mönnum varð ljóst að unnt væri skipta samsíðafrumsendunni út fyrir neitun sína og fá þannig aðrar gerðir af rúmfræði, sem eru jafn réttháar Evklíðsku rúmfræðinni.

Þær frumsendur sem við kynnum byggja lauslega á þeim frumsendum sem David Hilbert setti fram í byrjun 20. aldar í því markmiði að finna klassískri rúmfræði traustan rökfastan grunn.

Frumhugtök rúmfræðinnar eru punktar og línur. Í því felst að ekki er gerð tilraun til þess að skilgreina þessi hugtök heldur verða settar fram frumreglur sem lýsa þeim eiginleikum sem punkar og línur hafa. Punkar er venjulega táknaðir með stórum stöfum eins og $A,B,C,P$ og $Q$ en línur venjulega með littlum stöfum eins og $a,b,c,l$ og $m$. Önnur hugtök rúmfræðinnar eru svo skilgreind út frá þessum hugtökum.

Við gerum ráð fyrir að punktar geti legið á línum, þ.e. punktur $P$ getur legið á línu $l$ eða ekki. Einnig segjum við að $l$ liggi um $P$. Ef $l$ og $m$ eru tvær (ólíkar línur) og $P$ er punktur sem liggur á báðum, þá segjum við að $l$ og $m$ skerist í $P$ og $P$ sé skurðpunktur $l$ og $m$. Venslin að punktur liggi á línu nefnast leguvensl.

Nú getur það gerst að þrír punktar $A$, $B$ og $C$ liggi á línu $l$. Þessir punktar geta þá haft þann eiginleika að $B$ liggi á milli $A$ og $C$ (ekki endilega í miðjunni). Vensl sem fjalla um afstæða legu punkta á línu kallast raðvenls, því þau fjalla um röðun punkta, eins og t.d. punktar $A$, $B$, $C$ og $D$ liggja á línu $l$ í þessari röð. Til þess að tákna að $B$ liggi á milli punktanna $A$ og $C$ skrifum við $A*B*C$.

Að lokum þá hafa ólíkir punktar einhverja fjarlægð á milli sín. Það getur þá gerst að fjarlægðin milli $A$ og $B$ sé sú sama og fjarlæðin milli $C$ og $D$. Þetta skrifað $\linseg{A,B} \cong \linseg{C,D}$, lesið strikið $AB$ er samsniða strikinu $CD$ eða einfaldlega að strikin séu jafnlöng.

Frumsendunum flatarmyndarfræðinnar skiptum við í fimm flokka. Við stöldrum við eftir frumsendur hvers flokks og reynum að átta okkur á umtaki hans.

\subsection{Legufrumsendur}
Frumsendur þessa flokks fjalla um leguvenslin. Í þessum flokki eru þrjár frumsendur:
\begin{enumerate}
	\item Um tvo ólíka punkta liggur nákvæmlega ein lína.

	\item Á sérhverri línu liggja að minnska kosti tveir punktar.

	\item Til eru þrír punktar sem ekki liggja á sömu línu.
\end{enumerate}

Fyrsta frumsendan segir að tveir ólíkir punktar $A$ og $B$ ákvarði nákvæmlega eina línu. Við táknum þessa línu vanalega með

$$
\stline{A,B}.
$$

Ef línur $l$ og $m$ eru sama línan eða hafa engan punkt sameiginlegan þá segjum við að þær séu \emph{samsíða}, ritað $l\parallel m$. Ef $l$ og $m$ eru ólíkar línur þá geta þær ekki haft fleiri en einn sameiginlegan punkt, því ef $l$ og $m$ hafa ólíku punktana $P$ og $Q$ sameiginlega, þá er $l=\stline{P,Q}=m$ skv. fyrstu frumreglunni. Það er því ljóst að ef $l$ og $m$ eru ólíkar línur sem hafa punktinn $P$ sameiginlegan, þá er $P$ eini sameiginlegi punktur þeirra. Við segjum þá að $l$ og $m$ skerist í $P$. Fyrir tvær ólíkar línur $l$ og $m$ sem skerast, táknum við skurðpunkt þeirra

$$
l\wedge m.
$$

Hlutverk fyrstu frumsendunnar er því að tryggja að tveir punktar skilgreini ótvíræða línu, hlutverk annarrar frumsendurnnar er að tryggja að allar línur megi skilgreina með tveimur punktum. Að lokum er hlutverk þriðjum frumsendunnar að tryggja að til séu næginlega margir punktar til þess að til sé lína og ekki allir punktarnir liggi á henni. Við þurfum þrjá punkta svo unnt sé að tala um sléttu.

\subsection{Raðfrumsendur}

Frumsendur þessa flokks eru fjórar talsins. Þær fjalla um raðvenslin. Rifjum upp að við skrifum $A*B*C$ ef punktur $B$ liggur á milli $A$ og $C$. Við segjum einnig að punktar $A$ og $C$ séu \emph{hvor sínum megin} við punktinni $B$. Frumsendurnar eru:

\begin{enumerate}
	\item Ef $A*B*C$, þá eru $A,B$ og $C$ þrír ólíkir punktar á sömu línu, og $C*B*A$.

	\item Ef $A, B$ eru ólíkir punktar, þá er til að minnsta kosti einn punktur $C$ þannig að $A*B*C$.

	\item Ef $A*B*C$ þá gildir ekki $C*A*B$.

	\item Ef $A,B$ og $C$ eru þrír punktar sem liggja ekki á sömu línu og $l$ er lína sem gengur ekki í gegn um neinn punktanna $A, B$ eða $C$, en gengur í gegn um punkt á milli $A$ og $B$, þá er til punktur á $l$ sem liggur á milli $A$ og $C$ eða $B$ og $C$.
\end{enumerate}
